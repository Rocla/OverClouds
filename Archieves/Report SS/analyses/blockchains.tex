%-------------------------------------------------------
%    DOCUMENT CONFIGURATIONS
%-------------------------------------------------------

%-------------------------------------------------------
%    START OF BLOCKCHAINS ANALYSE
%-------------------------------------------------------
\subsection{Block-Chains}
\subsubsection{Worth it?}
It is important to note that with incoming quantum computers (predicted to appear wildly in 20ish years), the mining structure, the security and the anonymity must change from today's perspectives. 
As for today, the block-chain technology is at its hype, meaning that we see it has the best thing in the world.

However, from now the hype will decrease and maybe a new technology will emerge or/and the block-chains technology will evolve or be modeled to go in a direction we did not expect yet.

Yes, from today's perspective, the timeframe is pretty significant, it is worth the interest.

\subsubsection{What's next in crypto-currency?}

As of today, considering that the technology of block-chains will not change, and it still in use for crypto-currency, it could take two types of path.

One of the paths is the neverending death and birth of crypto-currencies. Indeed, once the mining is no more profitable, the security sharply decrease because miners are verifying the transactions and are playing the role of consensus for validating transactions. Miners are mining as long as the devices allow a profit (power consumption, device rentability, etc.). Best case scenario, the hardware technology continues to involve, as well as the required computational power. (Note that we are currently brute-forcing the solutions.) Moreover, based on the model of crypto-currency of Satoshi Nakamoto, Bitcoin, at some point in time, the maximum amount of coins will be reached, and the network will not generate coins (rewards) anymore. At this point, the only income of the miners will be the transaction fees. If the transactions fees are not high enough to motivate the miners to continue mining (and verifying/validation the operations), the currency will die due to the lack of security. So the miners will move to new profitable crypto-currency (note that they have a pretty advanced hardware for mining at this point, which will help them to start pretty well).

The second path is the modification of the source code of the actual crypto-currencies to make it compliant with the market evolution. For example, increase the maximum amount of coins, or make public keys quantum proof. Indeed, currently, the \textbf{ECDSA} is not quantum proof (however the hashing is at the moment, but SHA3 is ready, just to be safe). The problem is ECDSA, which during a transaction send the public key, and theoretically, a quantum computer can guess the private key from it. However, the address is still secure because it is the hashed public key.

However, the second path is \textbf{killing} the concept of a stable currency based an expendable raw material stock, and the social and economical results are pretty hard to define. A secondary question would be: What will happen, if tomorrow, we find a new gold mine, which holds the same amount of gold already retrieved (doubling by this mean the maximum quantity of gold available), and with a retrievable difficulty level a lot decreased, so it is again profitable to mine?

So, we do not know if it will be a next big crypto-currency. In my opinion, I would bet on Ethereum. However, again, it is personal.

\subsubsection{Predicted evolution in block-chains}
This subsection will be pretty short because at the moment, this technology is only starting to decedent the hype slope, and the only real evolution that pops out recently is the first version of  \textbf{Ethereum}\cite{VitalikButerin2013APlatform} (2013a) and more recently (2016) the Homestead version of Ethereum \cite{DR.GAVINWOOD2015ETHEREUM:DRAFT}.

The particularity of Ethereum is that use the currency as fuel to run smart contracts on the EVM (Ethereum Virtual Machine) using the power of each node on the network to do a calculation, and creating a consensus on the output. This technologies evolution has for example created a startup company named \textit{Slock.it} and an alternative "currency" \textit{DAO} (which is unmineable) that allows the IOT (internet of things) to interact with the crypto-currency Ether. It allows, for example, to control a lock, in a hotel, a door could be locked until a client paid the door to open.

\subsubsection{Proof-of-Work}
Read as PoW\cite{Dwork1993PricingMail, Jakobsson1999ProofsAbstract}. It is a protocol aiming to reduce the risks of DDOS attacks and family abuses by requiring that the client has done some computational work (processing time) before sending a request. It was a solution developed mainly for our financial world of transactions.

\subsubsection{Proof-of-Stake}
What is a stake? It is globally something that holds. In our case it is more like a flag can keep a land (a claimed property).

\paragraph{Proof-of-Stake?\cite{King2012PPCoin:Proof-of-Stake}}
Read as PoS. Usually in block-chains PoW, miners validate the transactions that came first depending on their CPU power. Note that the more CPU power you have (GPU, FPGA, ASIC, etc.) larger your influence is.

POS is the same thing but with different paradigms:

In one of them, Stakeholders validate with something they own (raw material like an internal currency). Moreover, put simple, everybody has a certain chance (proportional to the account’s balance) per amount of time of generating a valid raw material.

In another, we are not working with the amount of raw material owned, but with their age (for example, the raw material is multiplied by the time that it was unused) which gives a weighting factor. However with this paradigm, a collusion attack is pretty important, because we could have a super linearity by accumulating aged raw materials.

There are other different types of approaches but we will not details them all because they are not the best of consensus algorithms. (elitism, identity, excellence, storage, bandwidth, hash power, etc.)

\paragraph{Now, on the security side} By using raw material (sort of digital assets) defined by the consensus PoS avoids a Sybil\cite{G.LawrencePaulSundararaj1D.R.AnitaSofiaLiz22014Anti-SybilNetworks} attack. Which is is a technique where the attacker is trying to compromise a system by creating multiple duplicate or false identities. It is resulting into including false information, which then can mislead the system into making not intended decisions in favor to the attacker. By the way, PoW protects itself against a Sybil attack by using computational resources that exist extra-protocol.

However, in PoS' traditional approach, we have two major problems. The first is Nothing-at-Stake, and the second are Long range attacks.

\paragraph{Nothing-at-Stake}
The dominant problem is that smart nodes have no discouragement from being Byzantine\cite{Lamport1982TheProblem}. Indeed, signatures are very easy to produce, and they will not lose any tokens for being Byzantine. Another problem is that nodes with digital assets could never spend.

A solution to this would be to have a security layer on deposits, which would cancel Byzantine deposits. To achieve this, we would need to store information about nodes and their immoral behaviors (which are decided by the consensus) so the consensus would be able to punish them. Now, this works only if the transactions are not hidden (with the proof of malicious actions). Also, we should note that this security layer would ask more power for the consensus during the use of punished accounts. Slowing down the consensus is not acceptable because it acts as the authority and by this mean should be the cheaper to operate in power. Punishing the attackers with power consumption is fair.

Compared to PoW, where attackers are not receiving compensations for their computational power (which is a disincentive). The PoS' security layer is trying to disincentive attackers by removing their digital assets. It could be an interesting social experiment, however, in our human's economic point of view, attackers should be pretty well disincentivized.  

\paragraph{Long Range Attack}
In this type of attacks, the attacker controls account with no digital assets and is using them to create competing version of transactions. This attack is touching both traditional PoS, and the deposit security layer (as long as authentication ends in the genesis block).

The solution here would be to force nodes (and clients) to authenticate the consensus (for example with its state) by signing with the nodes that have something at stake currently, and nodes must have an updated list of nodes with deposits. It is usually called the \textit{weak subjectivity} method.

\paragraph{Ghost}
From the full name, Greedy Heaviest-Observed Sub-Tree protocol, introduced by Yonatan Sompolinsky and Aviv Zohar\cite{Sompolinsky2014AcceleratingChains}, it allows the PoW consensus to work with much lower latency than in the blockchain protocol from Satoshi Nakamoto \cite{SatoshiNakamoto2008Bitcoin:System}, and of course keeping it secure. Indeed, in blockchain based PoW, a miner is rewarded for each block found so the other miners can continue to mine on top of it. However, when a miner produces an orphaned block (a block that exists in the chain), they are not rewarded for their work, plus the consumed power was in vain because the work is unused by the consensus.

Here comes the solution, Ghost, which includes orphaned blocks. It introduces the notion of rewarding orphaned blocks to miners and increasing the security of the consensus with increased validations of a block in the block-chain.

\paragraph{Casper}
Now there is a friendly ghost in town; it is Casper\cite{Buterin2015UnderstandingCasper}. This protocol is based on Ghost, and must be integrated into Ethereum for the Serenity\cite{Buterin2014SlasherStake} version (final), however, the Metropolis version must go out before. We should also note that they released the Homestead version on 14th March 2016 (about a month before this Report release). It will work on the smart contracts.

\paragraph{Finally}
In comparison to PoW, PoS is much cheaper to secure, transactions speed is greater, and it is maybe the stepping stone into scaling the block-chain technology.

\subsection{Proof of Activity}
The protocol from Bentov, Lee, Mizrahi and Rosenfeld  \cite{Bentov2013ProofStake} which is implemented into PeerCoin (and its clones), is considered as a hybrid of the PoW and PoS. The nodes are doing PoW work by mining blocks and at the same time with the PoS (meaning that the block-chain includes both types of blocks).

\paragraph{The procedure}
\begin{itemize}
\item The PoW miner mine.
\item Block is found, the network is notified and creates a template. (multiple templates are possible)
\item The block hash is used to find random owners by using its hash as numbers to determine owners (nodes from the network).
\item Turn by turn each chosen owners sign the key with the key of the block.
\begin{itemize}
\item If a chosen owner is unavailable, the process paused. (it is not a problem this concurrently miners are still mining and generating new templates with different owners)
\end{itemize}
\item At some point in time, blocks will be signed, and the reward will be given to the miner and the owners.
\end{itemize}

\paragraph{Continues data exchange}
To reduce the data traffic, each template does not include a transaction list during the signing process itself; it is the last owner (signer) that is adding it when creating the block.

\subsubsection{Attacks on block-chains}
Block-chains is designed to be controlled by the consensus of nodes. It means that it can not be owned not controlled by a third party. Until now this goal has been pretty well achieved. However, experiences showed that the system is not perfect.

\paragraph{The 51\% attack}
It is the most interesting (in a social experiment point of view) and rewarding attack for the attacker. Indeed, if the attacker controls at least 51\% of the consensus, it is possible to manipulate transaction by validating malicious transactions. Pools owners can do this. Note that as it is today (for actives crypto-currencies), it is not more possible to mine on your own and be profitable, miners are forced to join pools and distribute the work and rewards between them. Meaning that it creates a vicious circle, the more miners are in a pool, the more power it has. The more power it has, the more reward are generated. Finally, this results in attracting, even more, miners because they also want a bigger and easier reward for mining, which leads to the security risk of malicious pool chief who will control the currency and the transactions. All around the internet people are always saying that it is dangerous, in fact, they are asking others to stop making a profit for the good of others, which is a human self-fish reasoning. Can't wait to see this case of a figure.

\paragraph{Spam attacks}
The idea here to make many transactions to the victim's wallet (by its addresses) and paralyze the legit transactions and by this way its incomes. Indeed, the network will have to process all the spam transactions as well as the legit transactions, meaning that the a delay is added before receiving the legit transactions. In some cases, like for Wikileaks\cite{TheBitcoinNews2015BitcoinAttacks}, which is depending on this kind of funding to live, it is pretty bad. Plus, since many transactions appear in a block, its value increase and miners will jump on it to get the reward, meaning that the legit transactions are but a bit behind because they have a lower reward. However, usually, the current crypto-currencies have an anti-spam solution. They have a minimum fee, and they increase the fee after each new transactions.

\paragraph{DDOS on exchange platforms}
The profit behind this type of attacks is to either steal wallets or ask a reason to release the servers. The crypto-currency is only affected by the depreciation of its value in "real" money because are not able to trade, and they are more likely to switch to another exchange platform or currency.

\paragraph{Special dedication to Mining malwares}
It is funny to see that hacking is evolving with the hype. Now instead of having zombie computers doing nothing waiting for DDOS attacks or whatever they are used to. They are now mining coins (generally connected to a pool). How smart is that? We think that it is amazing!

%-------------------------------------------------------
%    END OF BLOCKCHAINS ANALYSE
%-------------------------------------------------------