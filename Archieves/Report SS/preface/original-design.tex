%-------------------------------------------------------
%    DOCUMENT CONFIGURATIONS
%-------------------------------------------------------

%-------------------------------------------------------
%    START OF THE ORIGINAL DESIGN
%-------------------------------------------------------
\subsection{Original design}
During the project bootstrapping, global solutions and concepts emerged as the first overview of the project. Below is the result of the initial concept. However note that during the effective research and analyses, the project has evolved, sometimes resulting in new architectures and new recommendations.

\paragraph{The mesh of trust}
Acting like a consensus of trusted nodes on a node network.

\paragraph{Certificates} Each node has a unique nontransferable certificate, which can be regenerated as a new one. It is used to individualize nodes, and allow nodes to trust each other. The network is storing the certificates, and keep track of its proof-of-activity\cite{Bentov2013ProofStake} ranking.

Indeed, the certificate level increases over time by providing proof-of-activity. The level is also influenced by the amount of trust given to other nodes from the network. Note that nodes alone are not aware of others trust interactions, they kept informed of the sum of nodes willing to share their data with it.


\paragraph{Identities} Users can use any node from the network. For a user to identify himself, he needs to generate at least one unique identity. Identities are created from an existing node that is already part of the network. Each identity has a public and private key, a parent node (hardware of generation), and optional information (languages, avatar, name, etc.) for internal applications to the network.

An identity cannot be alternated. The user must generate a new one with the previous one was banned, and restart the process of acquiring trust from the network. Depending on situations, they must also need use another node.

\paragraph{Network requirements} A node does not need an identity to be able to connect to the network. It connects automatically and integrates the mesh of nodes by giving storage and computational power. However, an identity requires a node to connect to the network and interact with it. So, a node works without an identity, but an identity always needs a node.

\paragraph{Democracy} Identities can create votes and of course votes for submitted proposals, which are creations or modification of the rules ruling every transaction on the network. Each identity is weighted by default to the initial vote unit and can vote once. However, the consensus can decide to give more weight or votes to specific identities. By default, nodes can be used only once to give a vote (except if the consensus decides else-wise).

\paragraph{Bans} Certificates and identities can be banned from the network after a trial or directly from the consensus. A flag, seen globally by the network, is raised for the banished nodes or identities. During a trial, a random amount of random identities is asked to vote on the event that led to the trial.

\paragraph{Flags} Node can flag specific certificates or nodes on the network and apply rules on them, such as filters for minimum certificate level required to be able to connect to them as a relay. The flags are then used by the network, and route communications depending on the nodes' preferences.

For example, in the case of a node receiving data from another node that is not matching its filter, the data would be rejected at entrance. In the event of spam, which assumes that the attacker knows what destination (that he sees as void) to target, the target node will indeed consume power to deny requests. A solution has to be found for a case of a figure and prevent spam.

Another example, the node rerouting. A node can decide not to relay data. In this case, the data is either sent back, which can result in a load problem. Either the data is lost, which is in the event of a UDP-like protocol bad. In the case of a TCP-like protocol, the node would be searching for another node to go through. The second option could on another hand surcharge by searching new paths.

\paragraph{Storage} Data is stored across the network. It is spread on the network as encrypted chunks belonging to an identity, the private and public keys for the data is also spread across the network and belongs to the network. The data and its keys have redundancy chunks, also spread across the network.

\paragraph{Data owners} The owning rights are given and managed by the network to a specific identity. The owner can give as he wishes the reading, writing, and executing rights to any nodes or identity on the network. He can also set a public access for certificate or trust levels. The ownership is, of course, revocable or transferable by the consensus. However, the owner can also transfer the ownership to another identity. Both parties must accept the transfer. Note that he cannot transfer blamed data.

\paragraph{Blames} A node can anonymously blame chunks or nodes and leave a reason for the blame from the list of multilingual generic reasons. Note that the consensus can also create new categories. Each node can blame a specific data only once. A blaming identity is then linked to the note used to blame, by this mean it cannot vote more than once per data. Plus the first identity to use the node gets the blame validated.

Blames are telling the network that a node or a chunk is not appropriate to the other nodes. Multiple blames on a node may result in a ban of its certificate. Multiples blame on a specific chunk or chunks belonging to the same data resumes into a revocation of all rights given to nodes even the owner.

\paragraph{Data Tribunal} Once the required amount of blames reached the network select not related random identities and asks them to rate the blame. The digital judges would be able to read the reasons left during the blaming phase. However, they are not able to read the content of the data itself. Except is the consensus decides otherwise. Owners can ask for a second trial if they think the judgment was unfair and add a generic argument. For the second trial, different unrelated random nodes will have to rate the blame again. The decision from the second trial is definitive. The data is either archived, either the owner and related nodes get their rights back.

\paragraph{Internet Service Providers} They should not be able to interpret the network activity. They are only aware of encrypted tunnels made to random nodes. Like the Tor network, the gateway nodes never the same. It allows a censorship protection.

\paragraph{Internal crypto-currency} Nodes are automatically retrieving units of the internal crypto-currency with their proof-of-activity. The network owns the currency. Rewards are given to identities for good behavior and participation. The currency is transferable to unbanned identities. However, an identity has no interaction access to its balance during the trial phase. If the identity is banned, its currency balance is returned to the network. The use of the crypto-currency is not clear for the moment, but it could be used as a fuel like on Ethereum\cite{Ethereum2016EthereumDocumentation}.

\paragraph{Backdoor} The project is not friendly for third party authorities like governments, police investigations, companies, etc. However, the consensus has virtually unlimited power over the network and can model its democracy as it is pleased. We can easily imagine that the consensus decides to disclose all the pedophiles from the network with the goal that they would be punished by the \textit{real world}. The consensus decides what is right or wrong, and morality.

\paragraph{Artificial Intelligence} Based on today's technology it is unlikely to see a very smart AI emerge inside the project. However, we could imagine applications like self-preservation, meaning that the network could learn and decide by itself with the consensus is right or wrong.

\paragraph{Communication} The speed and size of the transactions should be compliant with any bandwidth. For example, the network should be able to work on a network of ALIX node bidden to a Xbee connection. Alternatively, networks from the emergent 3rd world.

\paragraph{3rd parties} Depending on the technologies used on the project, we could, in a first time, use third party services like Tor, Github, Twitter, BitTorrent trackers, etc. (note the public status and censorship of the example). The security could be compromised during the bootstrap phase in some cases because the ISP and others services could target and track nodes' activities.

\paragraph{Compromised users} In the case of malicious software presence on a system used by the node solutions can be taken into account. Assuming that there is less than X\% (exhausting value to estimate at this phase of the project) of malicious software designed for the network. It is indeed difficult to protect people from malicious people. We could add security layers, like a Pin or a passphrase, but if the user is running a keylogger, it will always be insecure. However, since the node can only be accessible from a unique hardware (certificate linked to the hardware), the stolen key could not be as useful as planned. However, the system could also be compromised by someone or something physical like the RUBBER USB\cite{HakshopRubberUSB}, in which case the hardware protection could be bypassed. Now we could think of security that only shows a virtual keyboard, but a specific program could sniff the mouse positions and actions. A solution could be to use a USB-key as key, but it could also be replicated if the key is writable. We could use an external hardware, but it would impact Overclouds' public attractivity. Moreover, in the end, how to be sure that the company making the encrypting hardware will not be hacked, resulting in the release of the algorithms for generating the keys? Another solution is to control the hardware and the software, assuming that there is no way for an external or internal entity to know what's the key. This last option would mean that Overclouds would have to be privatized somehow, and it would impact nature and vision of the project, by making part of the project proprietary.

\paragraph{Certificate or Identity Clones} The network does not allow clones using the network at the same time. The consensus will decide which one is the real one, and the clone will not be ignored by the network, no peers would accept a connection from it.

In the case an identity is stolen, in the current state of the concept, the attacker would have full access to the identity's assets. However, a solution to avoid this concern would be to link an identity to a unique node. Forcing the user to have a different identity for each node it connects to. The attacker would have to have a clone of the identity key and a clone of the node certificate, which increase the difficulty of the attack. However, it is still possible if the attacker has access to the original hardware and have the technical knowledge to emulate the hardware running the clone of victim's node certificate. To increase the security again and solve this problem, we would have to add an extra layer of security with an secure additional hardware such as a NitroKey\cite{NitrokeyNitokey}. Now, those solutions are considered extreme, in general, going this far into security is not necessary.  However, if those solutions enter into consideration, we could predict a substantial impact on new users willing to join the network.

Note that the consensus could be able to revoke a key or certificate. Protocol for asking the consensus to revoke an identity could be implemented. Alternatively, simply asking many people to blame the stolen identity or certificate.

\paragraph{Unanswered questions}
\begin{itemize}
\item Should we take into account that the wired Internet speed only improves?
\dots Probably \dots

\item Should we start from a programmable network such as Ethereum or MaidSafe? Alternatively, should we start from scratch?
\dots Make, buy or adapt study \dots

\item Would it be possible to allow nodes to run programs on the network like on Ethereum or with an API like on Maidsafe\cite{MaidSafe2014MaidSafe.netCommunity}.
\dots Probably, it could be interesting to make bots for content sharing, selling, etc. \dots

\item Is it possible to pledge that a node will be able to connect to the network with a random node? What about firewalls? Would connections only be based on TOR hidden service help? Alternatively, I2P? or Freenet? Alternatively, services as such? 
\dots The idea of only using a browser could be in jeopardy. \dots

\end{itemize}

%-------------------------------------------------------
%    END OF THE ORIGINAL DESIGN
%-------------------------------------------------------