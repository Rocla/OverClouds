%-------------------------------------------------------
%	DOCUMENT CONFIGURATIONS
%-------------------------------------------------------

%-------------------------------------------------------
%	START OF CRYPTOGRAPHY
%-------------------------------------------------------
\subsection{Cryptography}
\todo[inline]{TODO}

\subsubsection{Compare}

\begin{table}[!h]
\centering
\caption{Hashing 0-10MB Files /milliseconds based on \cite {Tarr2014PerformanceLibraries.}}
\label{hashing-0-10mb-files}
\begin{adjustbox}{center, width=\columnwidth-20pt}
\begin{tabular}{|l|l|l|l|l|}
\hline
Libraries & Sha1 (size) & Sha1 (hash) & Sha256 (size) & Sha256 (hash)	\\ \hline
sjcl				& - - -	-	& - - - -	& - - - -	& - - -	-	\\ \hline
crypto-js			& - - -		& - -		& -			& - - -		\\ \hline
forge				& +			& +			& + + +	+	& + + + +	\\ \hline
crypto-browserify	& + +		& + +		& + + +		& +	+ +		\\ \hline
crypto-mx           & null		& null		& +			& -	-		\\ \hline
git-sha1            & + + +		& + + +		& null		& null		\\ \hline
jshashes            & -			& -			& - -		& - - -		\\ \hline
rusha               & + + + +	& + + + +	& null		& null		\\ \hline
\end{tabular}
\end{adjustbox}
See Figures ~\ref{fig:hash-sha1}, ~\ref{fig:hash-ops-sha1}, ~\ref{fig:hash-sha256}, ~\ref{fig:hash-ops-sha256}
\end{table}

\begin{table}[!h]
\centering
\caption{Key Derivation (pbkdf2) based on \cite {Tarr2014PerformanceLibraries.}}
\label{key-derivation-pbkdf2}
\begin{adjustbox}{center, width=\columnwidth-20pt}
\begin{tabular}{|l|l|l|l|l|}
\hline
Libraries & Sha1 (time) & Sha1 (size) & Sha256 (time) & Sha256 (size)	\\ \hline
sjcl				& + + + +	& + + + +	& + + + + 	& + + + +		\\ \hline
crypto-js			& - - - -	& - - - -	& - - - -	& - - - -		\\ \hline
forge				& + + + +	& + +		& + + + +	& +				\\ \hline
crypto-browserify	& + + + +	& + +		& + + +		& - -			\\ \hline
\end{tabular}
\end{adjustbox}
See Figures ~\ref{fig:pbkdf2-sha1}, ~\ref{fig:pbkdf2-ops-sha1}, ~\ref{fig:pbkdf2-sha256}, ~\ref{fig:pbkdf2-ops-sha256}
\end{table}

\begin{table}[!h]
\centering
\caption{Hashing Small Files /milliseconds based on \cite {Tarr2014PerformanceLibraries.}}
\label{hashing-small-files}
\begin{adjustbox}{center, width=\columnwidth-20pt}
\begin{tabular}{|l|l|l|l|l|}
\hline
Libraries & Sha1 (size) & Sha256 (size) \\ \hline
sjcl				& - - -	& - - -		\\ \hline
crypto-js			& - -	& - -		\\ \hline
forge				& + +	& + + +		\\ \hline
crypto-browserify	& + + +	& + + + +	\\ \hline
crypto-mx           & null	& +			\\ \hline
git-sha1            & +		& null		\\ \hline
jshashes            & -		& -			\\ \hline
rusha               & + + + +	& null	\\ \hline
\end{tabular}
\end{adjustbox}
See Figures ~\ref{fig:small-hash-sha1}, ~\ref{fig:small-hash-sha256}
\end{table}

\begin{table}[!h]
\centering
\caption{Fastest Hashes /milliseconds based on \cite {Tarr2014PerformanceLibraries.}}
\label{hashing-small-files}
\begin{adjustbox}{center, width=\columnwidth-20pt}
\begin{tabular}{|l|l|l|l|l|}
\hline
Libraries & Sha1 (size) & Sha256 (size) & blake2s (size) \\ \hline
forge	& + + + +	& null		& null		\\ \hline
rusha	& null		& + + + +	& null		\\ \hline
blake2s	& null		& null		& + + + +	\\ \hline
\end{tabular}
\end{adjustbox}
See Figure ~\ref{fig:hash-ops-best}
\end{table}

\paragraph{blake2s} It is a new algorithm designed specifically for performance and is the fastest implementation. \textbf{rusha} is close behind it, and forge's \textit{sha256}.

Note that the above implementations display nearly a completely linear performance.

%-------------------------------------------------------
%	END OF COMMUNICATION
%-------------------------------------------------------