%-------------------------------------------------------
%    DOCUMENT CONFIGURATIONS
%-------------------------------------------------------

%-------------------------------------------------------
%    START OF STATE OF ART PAGE
%-------------------------------------------------------
\subsection{State of the Art}
\todo[inline]{TODO}

%-------------------------------------------------------
%    SIMILAR PRODUCTS SECTION
%-------------------------------------------------------
\subsubsection{Similar products (Existing Networks)}
\todo[inline]{TODO}

%-------------------------------------------------------
%    TRANSFER SECTION
%-------------------------------------------------------
\subsubsection{Transfer Protocols}
\todo[inline]{TODO}

%-------------------------------------------------------
%    PROTECTION SECTION
%-------------------------------------------------------
\subsubsection{Protection}
\todo[inline]{TODO}

%-------------------------------------------------------
%    CRYPTOGRAPHY SECTION
%-------------------------------------------------------
\subsubsection{Cryptography}
\todo[inline]{TODO}

%-------------------------------------------------------
%    HARDWARE SECTION
%-------------------------------------------------------
\subsubsection{Hardware}
\todo[inline]{TODO}

%-------------------------------------------------------
%    BLOCK-CHAINS SECTION
%-------------------------------------------------------
\subsubsection{Block-Chains}
\paragraph{So much hype in this, but what is it?}
Everybody relate it mainly to \textit{Bitcoin}\cite{SatoshiNakamoto2008Bitcoin:System}, indeed \textit{Bitcoin} introduced this technology (which is its main innovation), but \textit{Bitcoin} is not equivalent to chain-block.

Indeed, the main idea is that no one controls or owns the chain of blocks and forges a system for electronic transactions without relying on trust. A block contains a timestamp and information linking it to a previous block.

Note that blocks are sort of digital objects that record and confirm when and in what sequence transactions enter and are logged in the block chain. 

Blocks can be created by network users with specialized software and/or specifically designed equipment. Block creator are known as "miners" to reference the gold mining. Bitcoin showed us a pretty amazing evolution in the mining procedure, it was designed at the start by Satoshi Nakamoto for CPUs, then it quickly involved into GPU mining then into programmable (burnable) chips (VHDL) then now it goes into hard burned circuits.

In a cryptocurrency system, miners are incentivized to create blocks to collect two types of rewards: a pre-defined per-block award, and fees offered within the transactions themselves, payable to any miner who successfully confirms the transaction.
Every node in a decentralized system has a copy of the block chain. To avoid the need for a trusted third party to timestamp transactions, decentralized block chains use various timestamping schemes, such as proof-of-work.
\paragraph{Bitcoin \cite{SatoshiNakamoto2008Bitcoin:System}}
\todo[inline]{TODO}
\paragraph{Ethereum}
Seen today has the new way to use the block-chains technology. It uses the currency has fuel to execute turing-complete smart contracts. Contracts, see as autonomous agents, are programs that run on the Ethereum Virtual Machine. The EVM is being part of the protocol and runs on each client contributing to the network, they are all doing and storing the same calculations. Note that it's not really efficient to compute in parallel redundantly, but it offers a consensus for the computed results. 
\paragraph{Others}

%-------------------------------------------------------
%    DECENTRALIZATION SECTION
%-------------------------------------------------------
\subsubsection{Decentralized applications}
\todo[inline]{TODO}

%-------------------------------------------------------
%    REPUTATION SECTION
%-------------------------------------------------------
\subsubsection{Reputation Management}
\todo[inline]{TODO}

%-------------------------------------------------------
%    OS SECTION
%-------------------------------------------------------
\subsubsection{Operating Systems}
\todo[inline]{TODO}

%-------------------------------------------------------
%    TECHNOLOGIES SECTION
%-------------------------------------------------------
\subsubsection{Technologies}
\todo[inline]{TODO}

%-------------------------------------------------------
%    END OF STATE OF ART PAGE
%--------------------------------------------------------------------------------------------------