%-------------------------------------------------------
%    DOCUMENT CONFIGURATIONS
%-------------------------------------------------------

%-------------------------------------------------------
%    START OF BIG PICTURE PAGE
%-------------------------------------------------------
\subsection{Big Picture}
After being introduced to the project, let us present our current vision for the outcome of Overclouds. Note that the results of the big picture are not the bachelor project, allocated man-hours are clearly not enough. And of course, due to the early stage of the project, the vision may evolve. The bachelor work aims to do research and produce parts of the proposed final project.

\paragraph{Global}
The goal is to provide a product easy to use, technophobe-friendly, that does not require any special software or process running in the background. We are focusing on the browser experience; any browser should be able to run a node and be part of the network. The user should be able to use any hardware and main net (see our current internet) connections to join the network, very low bandwidth, very low computational power, Hyperboria\cite{HypeHyperboriaWhitepaper} support, or even be compatible with nonexistent technologies yet. The security allows to any user to be anonymous to ISP and censorship-resistant. Plus, any user could access its personal \textit{desktop} with its data right from any browser connected to the main net.

\paragraph{The consensus}
A consensus-driven network with unlimited power over the entire network and putting the digital democracy to the top. It works in harmony with the newly introduced concepts of \textit{Decentralized Storage}, \textit{Data Tribunal}, \textit{Mesh-of-Trust} and \textit{Community Contracts} which glues nodes, users' identities, digital contracts, storage, gateways, and transactions recording.

\paragraph{Network}
A serverless mesh of self-aware and smart nodes. Every transaction is verified and is mathematically elegant. Intelligent enough for self-preservation, and bypass consensus decisions if they are judged as self-destruction (for example, if the new minimal computational power is higher than the current maximum available). Noting that privacy and security is being applied to any layer of the network. However, a consensus decision can always bypass any rule, even a privacy rule. A registry name could be available to access the network easily from browser's URL (.over?) without using a gateway like overclouds.ch.

\paragraph{Storage}
The network is using cooperative nodes to store and process computational work. Data has many states, private, public, time to live, a countdown to go public, and versioning. Meaning that anonymous static and dynamic websites can also be hosted.

\paragraph{Crypto-Currency}
Consensus-driven currency, with an updatable genesis block by consensus. Nodes are automatically contributing to the network wealth. Coins are rewarded to identities by the consensus as a \textit{proof of participation}. It represents how an identity \textit{respects} and \textit{contributes} to the network, which could give him leadership powers like a stronger weight for votes in specific cases determined by the consensus and contribute even more to the network.

\paragraph{Self updating}
An ultimate achievement would be a self-programmed network driven by the consensus.\cite{Gabel2008DynamicallyDevelopers}

%-------------------------------------------------------
%    END OF BIG PICTURE PAGE
%-------------------------------------------------------
