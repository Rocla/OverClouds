%-------------------------------------------------------
%    DOCUMENT CONFIGURATIONS
%-------------------------------------------------------

%-------------------------------------------------------
%    START OF INTRODUCTION PAGE
%-------------------------------------------------------
\subsection{Introduction}
\paragraph{Today} The world and more particularly the digital world has an important concern for privacy. Indeed, Edward Snowden's revelations on NSA's massive spying\cite{EdwardSnowdenArchive} led to a media scandal. People started to feel that their digital privacy was at stake by the worldwide "spies", either the government, companies, or even unknown threats yet. The result of this fear induces the Internet community to evolve and to think about a new economical and political world.

\paragraph{New behaviors emerged} Right into the web users, a noticeable split has been made, into two classes of people. Individuals who do not know how to protect their privacy on the Internet, and people who do know how to protect their privacy. For the first group, some are not aware of the global privacy status; some don't care, and some don't know how to protect themselves. Concerning the second panel, we have some personal ethical problems, which led to this project.

\paragraph{The problem} It is understood that people want to avoid advertising companies' and governments' tracking. It is also known that some people are scared that their stolen private data could be exposed publicly. However, some private data must be sometimes public and in some cases forced to be made public. From our humanistic and unpolitical point of view, we believe, that the data from terrorist groups (including their members or partisans) have to be denounced, made public and requires that the data gets locked down into a secured digital safe. Why archiving and not just deleting them? We, again, believe that regardless how bad the data is (as it is seen today), it is still part of our history, and it is our duty to preserve them as a legacy for the future generations to understand and learn from our present mistakes.

\paragraph{Intellectual Properties} Read as IP. Our social and political evolution led to copyright laws, which prohibit the free sharing of any data such as art and knowledge by default. Indeed, our cultures specify that any intellectual discovery or advancement is by default protected by copyrights (we will not discuss the process here either the use of the copyright system). All this usually leads to knowledge or data retention, which makes us, the authors of this project, feel that the IP system, is a serious threat for our humanity legacy.

\paragraph{Economy} However, it is also understood that with our economic system, people need to make a living out of the IP, which leads into higher protection for their data. Our digital methods of protection are to encrypt the data with a digital key provided by the owner (generally companies). The encrypted data is then stored in data centers or on hard-drives (which could sometimes be bad for the long-term preservation). Owners then have the choice of sharing their (copyrighted) material and are usually also choosing to use encrypted manners to distribute them. We are also protecting the transmission to restrain people from looking at them without owners' consent and redistributing them with consent (money is in most of the cases involved). Therefore, the data is protected during storage and transmission, and we also have a layer above that allows only to decrypt the piece of data a user is looking at on the fly via DRM's technologies. Technically, the data is never totally in clear anymore.

\paragraph{Game of cat and mouse} Some people are protecting data, other are trying to break the protection to retrieve the data in clear. Moreover, they both use encryption technologies to secure their transmissions. Meaning that the owners are protecting their data, and the \textit{thieves} are also protecting the stolen data, to not be caught. In the end, the same data is being protected and transmitted, however, never in clear. This procedure makes us believe that something is wrong with the system and that it could be considered as a threat for the humanity global knowledge and culture.

\paragraph{Legacy} We are confronted with another ethical problem. We are entering into an entirely encrypted data era, which will make all our data appear as a bunch of random noises, which we estimate on a personal point of view that for various reasons they are bad for us. We are thinking about how \textbf{all} our global knowledge is and will be sorted shortly. Data such as art or knowledge, how will our next generations of humans be able to retrieve it in a few years? How can we be sure that the unbreakable keys (quantum proof for example) that we are aiming to make today will not be lost and so the data with it?
Now let's think about the idea of randomizing (adding noises) our communication signals that we are sending into outer space. The signals that we are sending out to space are already today much different from what it looked like in the sixties. For example, purely analog signals (TV, FM radios) are depreciated and got replaced with a digital transmission (DVB, DAB), and of course, most of them are encrypted (pay-to-watch tv, pay-to-listen radios). Moreover, our communication technologies evolve into higher frequencies and always fewer consumption requirements. All this is resulting from the fact that today, Earth radiates probably much fewer waves than before, and it is seen as random noises.

\paragraph{Taking the wrong direction} It is easily imaginable that soon a company will say: "Hey, we are specialized in transmitting encrypted data, nobody will ever know who you are, what you are doing, and when you are doing it. All we have to do is to buy from us, with a crypto-currency, a key every month.". See this as an evolution of the anonymity services such as VPN or TOR\cite{TorTor}. We do not have a sharp opinion on either it is a good or a bad thing. However, it will create a privatization of the data security, and it is most likely to destroy any hopes of retrieving the encrypted data in the future.

\paragraph{Overclouds} The project aims into a different approach for protecting people's privacy and ensure a legacy for future generations.
The main idea is that Overclouds is a consensus-driven anonymous network of nodes with storage and computation power. Each node is aware of other nodes, but they are unable by default to identify its owner on the main net (whatever the physical technology of communication at any given point of time). The consensus has limitless power over the network. It can, for example, decide to disclose the physical location of particular nodes, if they are considered malicious or wrong for the rest of the network (such as terrorists, scammers, black hats, and so on).

\paragraph{Nice to have} We also would like Overclouds to be designed not act like random noises. Preferably, external and internal listener/watcher should see a nice mathematically driven signal. The mathemagics behind would allow any source node to predict exactly how the consensus will handle its data. However, it should be unpredictable for other nodes, but still elegant to look at.

%-------------------------------------------------------
%    END OF INTRODUCTION PAGE
%-------------------------------------------------------
